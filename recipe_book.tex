\documentclass{tufte-book}

\hypersetup{colorlinks}% uncomment this line if you prefer colored hyperlinks (e.g., for onscreen viewing)

%%
% Book metadata
\title{A Cookbook\thanks{Thanks to my mother.}}
\author[Ryan Orendorff]{Ryan Orendorff}
% \publisher{Publisher of This Book}

%%
% If they're installed, use Bergamo and Chantilly from www.fontsite.com.
% They're clones of Bembo and Gill Sans, respectively.
%\IfFileExists{bergamo.sty}{\usepackage[osf]{bergamo}}{}% Bembo
%\IfFileExists{chantill.sty}{\usepackage{chantill}}{}% Gill Sans

%\usepackage{microtype}

%%
% Just some sample text
\usepackage{lipsum}

%%
% For nicely typeset tabular material
\usepackage{booktabs}

%%
% For graphics / images
\usepackage{graphicx}
\setkeys{Gin}{width=\linewidth,totalheight=\textheight,keepaspectratio}
\graphicspath{{graphics/}}

% The fancyvrb package lets us customize the formatting of verbatim
% environments.  We use a slightly smaller font.
\usepackage{fancyvrb}
\fvset{fontsize=\normalsize}

%%
% Prints argument within hanging parentheses (i.e., parentheses that take
% up no horizontal space).  Useful in tabular environments.
\newcommand{\hangp}[1]{\makebox[0pt][r]{(}#1\makebox[0pt][l]{)}}

%%
% Prints an asterisk that takes up no horizontal space.
% Useful in tabular environments.
\newcommand{\hangstar}{\makebox[0pt][l]{*}}

%%
% Prints a trailing space in a smart way.
\usepackage{xspace}

\newcommand{\TL}{Tufte-\LaTeX\xspace}

% Prints the month name (e.g., January) and the year (e.g., 2008)
\newcommand{\monthyear}{%
  \ifcase\month\or January\or February\or March\or April\or May\or June\or
  July\or August\or September\or October\or November\or
  December\fi\space\number\year
}


% Prints an epigraph and speaker in sans serif, all-caps type.
\newcommand{\openepigraph}[2]{%
  %\sffamily\fontsize{14}{16}\selectfont
  \begin{fullwidth}
  \sffamily\large
  \begin{doublespace}
  \noindent\allcaps{#1}\\% epigraph
  \noindent\allcaps{#2}% author
  \end{doublespace}
  \end{fullwidth}
}

% Inserts a blank page
\newcommand{\blankpage}{\newpage\hbox{}\thispagestyle{empty}\newpage}

\usepackage{units}

% Typesets the font size, leading, and measure in the form of 10/12x26 pc.
\newcommand{\measure}[3]{#1/#2$\times$\unit[#3]{pc}}

% Macros for typesetting the documentation
\newcommand{\hlred}[1]{\textcolor{Maroon}{#1}}% prints in red
\newcommand{\hangleft}[1]{\makebox[0pt][r]{#1}}
\newcommand{\hairsp}{\hspace{1pt}}% hair space
\newcommand{\hquad}{\hskip0.5em\relax}% half quad space
\newcommand{\TODO}{\textcolor{red}{\bf TODO!}\xspace}
\newcommand{\ie}{\textit{i.\hairsp{}e.}\xspace}
\newcommand{\eg}{\textit{e.\hairsp{}g.}\xspace}
\newcommand{\na}{\quad--}% used in tables for N/A cells
\providecommand{\XeLaTeX}{X\lower.5ex\hbox{\kern-0.15em\reflectbox{E}}\kern-0.1em\LaTeX}
\newcommand{\tXeLaTeX}{\XeLaTeX\index{XeLaTeX@\protect\XeLaTeX}}
% \index{\texttt{\textbackslash xyz}@\hangleft{\texttt{\textbackslash}}\texttt{xyz}}
\newcommand{\tuftebs}{\symbol{'134}}% a backslash in tt type in OT1/T1
\newcommand{\doccmdnoindex}[2][]{\texttt{\tuftebs#2}}% command name -- adds backslash automatically (and doesn't add cmd to the index)
\newcommand{\doccmddef}[2][]{%
  \hlred{\texttt{\tuftebs#2}}\label{cmd:#2}%
  \ifthenelse{\isempty{#1}}%
    {% add the command to the index
      \index{#2 command@\protect\hangleft{\texttt{\tuftebs}}\texttt{#2}}% command name
    }%
    {% add the command and package to the index
      \index{#2 command@\protect\hangleft{\texttt{\tuftebs}}\texttt{#2} (\texttt{#1} package)}% command name
      \index{#1 package@\texttt{#1} package}\index{packages!#1@\texttt{#1}}% package name
    }%
}% command name -- adds backslash automatically
\newcommand{\doccmd}[2][]{%
  \texttt{\tuftebs#2}%
  \ifthenelse{\isempty{#1}}%
    {% add the command to the index
      \index{#2 command@\protect\hangleft{\texttt{\tuftebs}}\texttt{#2}}% command name
    }%
    {% add the command and package to the index
      \index{#2 command@\protect\hangleft{\texttt{\tuftebs}}\texttt{#2} (\texttt{#1} package)}% command name
      \index{#1 package@\texttt{#1} package}\index{packages!#1@\texttt{#1}}% package name
    }%
}% command name -- adds backslash automatically
\newcommand{\docopt}[1]{\ensuremath{\langle}\textrm{\textit{#1}}\ensuremath{\rangle}}% optional command argument
\newcommand{\docarg}[1]{\textrm{\textit{#1}}}% (required) command argument
\newenvironment{docspec}{\begin{quotation}\ttfamily\parskip0pt\parindent0pt\ignorespaces}{\end{quotation}}% command specification environment
\newcommand{\docenv}[1]{\texttt{#1}\index{#1 environment@\texttt{#1} environment}\index{environments!#1@\texttt{#1}}}% environment name
\newcommand{\docenvdef}[1]{\hlred{\texttt{#1}}\label{env:#1}\index{#1 environment@\texttt{#1} environment}\index{environments!#1@\texttt{#1}}}% environment name
\newcommand{\docpkg}[1]{\texttt{#1}\index{#1 package@\texttt{#1} package}\index{packages!#1@\texttt{#1}}}% package name
\newcommand{\doccls}[1]{\texttt{#1}}% document class name
\newcommand{\docclsopt}[1]{\texttt{#1}\index{#1 class option@\texttt{#1} class option}\index{class options!#1@\texttt{#1}}}% document class option name
\newcommand{\docclsoptdef}[1]{\hlred{\texttt{#1}}\label{clsopt:#1}\index{#1 class option@\texttt{#1} class option}\index{class options!#1@\texttt{#1}}}% document class option name defined
\newcommand{\docmsg}[2]{\bigskip\begin{fullwidth}\noindent\ttfamily#1\end{fullwidth}\medskip\par\noindent#2}
\newcommand{\docfilehook}[2]{\texttt{#1}\index{file hooks!#2}\index{#1@\texttt{#1}}}
\newcommand{\doccounter}[1]{\texttt{#1}\index{#1 counter@\texttt{#1} counter}}

% Generates the index
\usepackage{makeidx}
\makeindex

\begin{document}

% Front matter
\frontmatter

% r.1 blank page
\blankpage

% v.2 epigraphs
\newpage\thispagestyle{empty}
\openepigraph{%
Delicous
}{Ryan Orendorff%, {\itshape A Cookbook}
}
\vfill
\openepigraph{%
Awesome
}{Rhino}
\vfill
\openepigraph{%
OMG ROFLCOPTER
}{Ryry}


% r.3 full title page
\maketitle


% v.4 copyright page
\newpage
\begin{fullwidth}
~\vfill
\thispagestyle{empty}
\setlength{\parindent}{0pt}
\setlength{\parskip}{\baselineskip}
Copyright \copyright\ \the\year\ \thanklessauthor

% \par\smallcaps{Published by \thanklesspublisher}

\par\smallcaps{tufte-latex.googlecode.com}

\par This cookbook is a free foodgram: you can redistribute it and/or modify
    it under the terms of the {\allcaps GNU} General Public License as published by
    the Free Software Foundation, either version 3 of the License, or
    (at your option) any later version.

    This foodgram is distributed in the hope that it will be useful,
    but {\allcaps WITHOUT ANY WARRANTY}; without even the implied warranty of
    {\allcaps MERCHANTABILITY} or {\allcaps FITNESS FOR A PARTICULAR PURPOSE}.  See the
	{\allcaps GNU} General Public License for more details.

    To obtain a copy of the {\allcaps GNU} General Public License, please visit \url{http://www.gnu.org/licenses/}.

\par\textit{First printing, \monthyear}
\end{fullwidth}

% r.5 contents
\tableofcontents

% r.7 dedication
\cleardoublepage
~\vfill
\begin{doublespace}
\noindent\fontsize{18}{22}\selectfont\itshape
\nohyphenation
Dedicated to my mother.
\end{doublespace}
\vfill
\vfill

%%
% Start the main matter (normal chapters)
\mainmatter


\chapter{Appetizers}

\section{Mexican Dip}\index{Mexican Dip}

\marginnote[25pt]{%
	\fontfamily{ppl}\selectfont
	\begin{tabular}{ll}
	    \toprule
		Cream Cheese & \unit[1]{package} \\
		Refried Beans & \unit[15]{oz. can} \\
		Salsa & \unit[15]{oz. jar} \\
		Green Onions (chopped) & \unit[\nicefrac{3}{4}]{cup} \\
		Cheddar Cheese (grated) & \unit[1\nicefrac{1}{2}]{cups} \\
		Tortilla Chips & \\
  		\bottomrule
  \end{tabular}
}
\newthought{Spread cream} cheese evenly over the bottom of a 9x12 inch pan. Spread refried beans on top of the cream cheese. Sprinkle green onions on top of the refried beans, press down into beans. Pour jar of salsa over the onions and refried beans. Sprinkle cheddar cheese over the entire pan. Heat in the oven at 350 degrees for approximately 20 minutes. Serve with tortilla chips. \footnote{Thanks mom for the recipe.}

\begin{quote}
I honestly have no idea where this recipe came from. Our family is Ukrainian Canadian, and there are not a lot of beans in either place. Either way, this is an easy to make appetizer for a party or dinner for the family that tastes great and can be stored easily if there are left overs. If you are not satisfied, \emph{change your salsa or chips!}
\end{quote}



\chapter{Main Course}

\section{Cream Chicken}\index{Cream Chicken}

\marginnote[25pt]{%
	\fontfamily{ppl}\selectfont
	\begin{tabular}{ll}
	    \toprule
		Chicken & \unit[4]{ breasts} \\
		\midrule
		Butter & \unit[\nicefrac{1}{4}]{pound} \\
		Onion & \unit[1]{} \\
		Whipping Cream & \unit[1]{pint} \\
		Sour cream & \unit[\nicefrac{1}{8}]{ounce} \\
  		\bottomrule
  \end{tabular}
}

\newthought{Cook four} chicken breasts for 45 minutes to an hour until cooked,
can be boneless or with skin and bone, if with skin and bone, you need to take
all breast meat off and discard skin and bones.  \footnote{Thanks mom for the
recipe.}

\newthought{For the sauce}, Saute 1/4 pound of butter with one onion until onion
is clear at low heat, making sure that butter does not brown.  Add in one pint
of whipping cream, bring to soft boil constantly stirring. Once whipping cream
and onion mixture is at a soft boil add in \nicefrac{1}{8} ounce container of sour cream.  Stir to mix, making sure that it \emph{does not} boil. If it boils the mixture can curdle.  Add chicken back in to cream sauce mixture, heating through.

\newthought{Serve} with mashed potatoes.


%% FIXME: Add cream chicken note.
\begin{quote}
It is delicious. Just \emph{do it}.
\end{quote}


\chapter{Desserts}

\section{Nanaimo Squares}\index{Nanaimo Squares}


\marginnote[25pt]{%
	\fontfamily{ppl}\selectfont
	\begin{tabular}{ll}
	    \toprule
	    Butter (soft) & \unit[\nicefrac{1}{2}]{cup} \\
	    Sugar (white) & \unit[5]{tbsp.} \\
	    Cocoa & \unit[5]{tbsp} \\
	    Egg (beaten) & \unit[1]{} \\
	    Vanilla & \unit[1]{tsp.} \\
		\midrule
		Graham Wafer Crumbs & \unit[2]{cups} \\
		Coconut	& \unit[1]{cup} \\
		Walnuts (chopped) & \unit[\nicefrac{1}{2}]{cup} \\
		\midrule
		Icing Sugar & \unit[3]{cups} \\
		Butter	& \unit[6]{tbsp.} \\
		Milk & \unit[4\nicefrac{1}{2}]{tbsp.} \\
		Vanilla Custard Powder & \unit[3\nicefrac{3}{4}]{tbsp.} \\
		\midrule
		Chocolate Chips & \unit[\nicefrac{1}{2}]{bag} (4\,oz)\\
		Butter & \unit[1]{tbsp.} \\
  		\bottomrule
  \end{tabular}
}
\newthought{Place first} five ingredients in a pot.  Stir until butter is melted (make sure you don't cook too long).  Mix next three ingredients and add to first mixture.  Pack into ungreased 9x9'' pan.  Chill in fridge while you are making the next layer. Combine next four ingredients, cream and spread over mixture in pan.  (You'll have to beat this extra hard - if you don't have an automatic mixer.)  Chill until firm.  Melt chocolate and butter and spread evenly over top of chilled mixture.  Cut into bars about \unit[1\nicefrac{1}{2}]{inch} inch by \unit[1\nicefrac{1}{2}]{inch}.

\begin{quote}
This dessert originated in Ladysmith, just south of Nanaimo on Vancouver Island in the early 1950s. Local stories tell of a housewife, Mabel Jenkins, who submitted the recipe for a fundraiser\footnote{\url{http://en.wikipedia.org/wiki/Nanaimo_bar}}. I can remember seeing them at the M\&M meat shops (because chocolate is a form of meat) and Tim Hortons. I still attest that my mother's recipe made the best treats, so now you too can experience the expertise that only a mother with a very hungry boy can master.
\end{quote}

\section{Bananas Foster}\index{Bananas Foster}

\marginnote[25pt]{%
	\fontfamily{ppl}\selectfont
	\begin{tabular}{ll}
	    \toprule
		Bananas (ripe, firm) & \unit[4]{} \\
		Sugar (dark brown) & \unit[\nicefrac{3}{4}]{cup} (packed) \\
	    Butter (unsalted) & \unit[6]{tbps.} \\
		Dark Rum & \unit[\nicefrac{1}{4}]{cup} \\
		Vanilla Extract & \unit[1]{tsp.} \\
		Cinnamon & \unit[\nicefrac{3}{4}]{tsp.} \\
		Vanilla Ice Cream & \\
  		\bottomrule
  \end{tabular}
}
\newthought{In a large} saut\'{e} pan over medium heat, combine the butter, brown sugar and cinnamon. Cook, stirring occasionally, until the butter melts and the sugar is dissolved, 4 to 5 minutes. Add the banana slices (peeled, halved crosswise and then lengthwise (into quarters)) and cook, turning once, until just tender, 1 to 2 minutes per side.

In a small cup, stir together the rum and vanilla. Turn off the burner, then add the rum mixture to the banana mixture. Using a long match or lighter, light the alcohol by placing the flame just inside the outer edge of the pan. Stand as far back from the pan as possible, keeping your face and hands away from the pan. The flame will be a faint blue but will be very hot. It should extinguish in 5 to 10 seconds. Holding the pan handle with an oven mitt, gently shake the pan from side to side to coat the bananas with the sauce.
Scoop vanilla ice cream into individual bowls. Spoon some of the bananas and sauce over the ice cream, dividing evenly. Serve immediately. Serves 4 to 6.

\begin{quote}
\emph{``Flamb\'{e}ed desserts have long illuminated historic New Orleans restaurant dining rooms with their showstopping flames''}\footnote{\url{http://www.williams-sonoma.com/}}. Apparently. I just remember watching this for the first time at a restraunt as a chef lit a pan on fire and poured the result onto my ice cream. Being a pyromaniac, this seemed like the best dessert ever fathomed, so I naturally included it into this cookbook so others could wow and delight their guest. No guarantee on fire safety, though. Be careful. :-)
\end{quote}

\section{Ice Cream Dessert}\index{Ice Cream Dessert}

\marginnote[25pt]{%
	\fontfamily{ppl}\selectfont
	\begin{tabular}{ll}
	    \toprule
	    Special K Cereal & \unit[6]{cups} \\
		Vanilla Ice Cream & \unit[1]{quart} \\
		Hershey's Chocolate Syrup & \unit[1]{can} \\
		Coconut & \unit[1]{cup} \\
		Butter (melted) & \unit[\nicefrac{2}{3}]{cup} \\
		Sugar (brown) & \unit[\nicefrac{1}{2}]{cup} \\
  		\bottomrule
  \end{tabular}
}
\newthought{Mix cereal}, coconut, brown sugar and melted butter together in a large bowl. Spread \nicefrac{1}{2} of this mixture on the bottom of a 9x12 pan. Cut vanilla ice cream into 1 to \unit[\nicefrac{11}{4}]{inch} slices and place on the cereal mixture spread out so that it covers the entire pan evenly. Cover the ice cream with most of the tin of the chocolate syrup. Spread the remaining cereal mixture over the top of the syrup. Drizzle remaining chocolate syrup over the top to decorate. Freeze for a minimum of two hours. Cut into large squares. \footnote{Thanks to my mother for the recipe.}

\begin{quote}
My mother has made this dessert since I was a very small child -- and probably before, but I choose to believe there was no such time. It has always brought me great joy to help her make it, and of course to help her eat the results. This dessert makes me happy just thinking about it, and I hope you will experience the same joy as well.
\end{quote}

\section{Chocolate Macaroons}\index{Chocolate Macaroons}

\marginnote[25pt]{%
	\fontfamily{ppl}\selectfont
	\begin{tabular}{ll}
	    \toprule
		Oats (rolled) & \unit[4\nicefrac{1}{4}]{cups} \\
		Sugar & \unit[2]{cups} \\
		Coconut (dried or flaked) & \unit[1\nicefrac{3}{4}]{cups} \\
		Milk & \unit[5]{fl. oz.} \\  % 5/8 cup
		Butter & \unit[\nicefrac{1}{4}]{lb.} \\  % half a cup
		Cocoa Powder (unsweetened) & \unit[\nicefrac{1}{3}]{cup} \\
		Vanilla Extract & \unit[\nicefrac{1}{2}]{tsp.} \\
		Salt & \unit[\nicefrac{1}{4}]{tsp.} \\
  		\bottomrule
  \end{tabular}
}
\newthought{In a large} bowl, mix together the rolled oats and the coconut; set aside. In a large pot, place butter, milk, sugar, vanilla, salt, and cocoa powder. Bring these to a boil over low heat, stirring occasionally. Remove the pot from the stove and stir in the mixture of oats and coconut. With a large serving spoon or ice cream scoop, form round macaroons on a very lightly greased cookie sheet, flattening them slightly (a little oil on the spoon will help keep the mixture from ticking). Allow to cool at room temperature for at least 30 minutes. Yields 12 large cookies.\footnote{Mom again.}

\begin{quote}
	I don't know where to begin with this one. It is delicious. It is absolutely not nutritious. It has a lovely coconut flavor, and even people who do not like oatmeal, or other oat derivatives, will enjoy this treat. It even has a dirty word nick name (can you guess it?). Easy, tasty, chocolate, coconut. What more could one ask for?
\end{quote}

\backmatter

% \bibliography{sample-handout}
% \bibliographystyle{plainnat}


\printindex

\end{document}

